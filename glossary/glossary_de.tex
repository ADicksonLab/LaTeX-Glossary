%%%%%%%%%%%%%%%%%%%%%%%%%%%%%%%%%%%%%%%%%%%%%%%%%%%%%%%%%%%%%%%%
% BEGIN GLOSSARY_DE
%%%%%%%%%%%%%%%%%%%%%%%%%%%%%%%%%%%%%%%%%%%%%%%%%%%%%%%%%%%%%%%%

% A
% B
\newglossaryentry{BackingBean}
{
	name=Backing Bean, 
	description={wird im Zusammenhang mit JavaServer Faces verwendet und bildet eine Java Klasse auf dem Server die durch das Verhalten von Komponenten auf der Weboberfläche beispielsweise mit Werten befüllt wird}
}

\newglossaryentry{Bro}{
	name=Bro,
	description={bezeichnet sich als Network Security Monitor, eine Art \acrfull{IDS}, welches Netzwerkdaten ließt und in Form von Events weitergibt. Diese Events können in eigenen bro-Skripts verarbeitet und interpretiert werden. Für bestimmte Situationen können log Dateien erzeugt werden, wie für den Fall, dass ein unbekannter Host eine Aktion ausführt, die er nicht soll}
}

\newglossaryentry{Bug}
{
	name=Bug,
	description={ist ein Fehler oder ein unerwünschtes Verhalten einer Software},
	plural=Bugs
}

% C
% D
\newglossaryentry{d3js}{
	name={D3},
	description={steht für \emph{Document-Driven Documents} und ist eine JavaScript Bibliothek mit der Daten graphisch dargestellt werden können. Dazu werden \acrshort{SVG} Elemente im \acrshort{DOM} erstellt und diese mit \acrshort{CSS} angepasst}
}

% E
\newglossaryentry{elasticsearch}{
	name=Elasticsearch,
	description={ist eine Datenbank zur Volltextsuche die eine \acrfull{REST} Schnittstelle zum Abfragen der Daten bietet. Die Anfragen sowie die Ergebnisse werden als \acrfull{JSON} übertragen}
}
% F
\newglossaryentry{ForkGit}{
	name=Fork,
	description={beschreibt in Git das kopieren eines Projektes. Dabei wird das daraus entstehende Projekt getrennt von dem Original weiterentwickelt}
}

% G
\newglossaryentry{Git}{
	name=Git,
	description={ist ein Programm zur Versionsverwaltung von Quelltext}
}

%TODO: Differenz zwischen zwei Zeitpunkten???
\newglossaryentry{Graphite}{
	name=Graphite,
	description={ist ein System zum Speichern von Datenserien mit dem Fokus darauf, die Daten über einen länger Zeitraum zu speichern und mit bereitgestellten Funktionen zu verarbeiten. So ist es beispielsweise möglich mit einer \emph{derivative} Funktion die Ableitung einer Datenserie zwischen zwei Zeitpunkten um deren Differenz (delta) zu bestimmen}
}
% H
% I
% J
\newglossaryentry{JavaException}{
	name=Exception,
	description={ist ein Fehler in einem Programm der durch den Programmierer oder eine unerlaubte Aktion ausgelöst und an übergeordnete Instanzen weitergeleitet wird. Diese kann an einer höheren Stelle gefangen und behandelt werden}
}

\newglossaryentry{JavaEnterpriseEdition}{
	name=Java Enterprise Edition,
	description={beschreibt eine Sammlung von Spezifikationen und Schnittstellen welche meist in einem Client-Server Szenario Serverseitig ausgeführt werden}
}

% K
% L
% M
\newglossaryentry{Margin}{
	name=Margin,
	description={definiert den Abstand außerhalb eines Elements}
}

%TODO: Translate
\newglossaryentry{maven}{
	name={Maven},
	description={is a software project management and comprehension tool. Based on the concept of a project object model (POM), Maven can manage a project's build, reporting and documentation from a central piece of information}
}

% N
% O
% P
\newglossaryentry{Padding}{
	name=Padding,
	description={definiert den Abstand zwischen der grenze eines Elementes und dessen Inhalt}
}

\newglossaryentry{ProtoBuf}{
	name=Protocol Buffer,
	description={ist eine von Google zur Verfügung gestellte Bibliothek für den Austausch von Nachrichten über das Netzwerk}
	plural=Protocol Buffers
}

% Q
% R
\newglossaryentry{Release}{
	name=Release,
	description={beschreibt eine abgeschlossene Softwareversion}
}

% S
% T
%TODO: Translate (source: https://www.wireshark.org/docs/man-pages/tshark.html)
\newglossaryentry{tshark}{
	name=tshark,
	description={is a network protocol analyzer. It lets you capture packet data from a live network, or read packets from a previously saved capture file, either printing a decoded form of those packets to the standard output or writing the packets to a file. TShark's native capture file format is pcap format, which is also the format used by tcpdump and various other tools}
}
% U
% V
% W
\newglossaryentry{Wireshark}{
	name=Wireshark,
	description={is the world's foremost network protocol analyzer. It lets you see what's happening on your network at a microscopic level. It is the de facto (and often de jure) standard across many industries and educational institutions}
}
% X
% Y
% Z